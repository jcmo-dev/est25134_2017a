%
%
\documentclass[11pt]{article}

%   --------------------------------
%               Packages
%   --------------------------------
\usepackage{algorithm}
\usepackage{algorithmic}
\usepackage{amsmath}
\usepackage{amssymb}
\usepackage{amsthm}
\usepackage[spanish]{babel}
\usepackage[sans]{dsfont}
\usepackage[left=3cm,top=3cm,bottom=3cm,right=3cm]{geometry}
\usepackage[pdftex]{graphicx}
\usepackage[scaled=0.92]{helvet}
\usepackage{color}
\usepackage[usenames,dvipsnames,svgnames,table]{xcolor}
\usepackage[colorlinks=true,
            linkcolor=red,
            urlcolor=blue,
            citecolor=gray]{hyperref}
\usepackage[latin1]{inputenc}
\usepackage{listings}
\usepackage{longtable}
\usepackage{lscape}
\usepackage{mathrsfs}
\usepackage[round]{natbib}
\usepackage{subfigure}
\usepackage{supertabular}
\usepackage{syntonly}
\usepackage{tabularx}
\usepackage{titlesec}
\usepackage{url}
\usepackage{verbatim}

%   --------------------------------
%               Directories
%   --------------------------------
\graphicspath{{../figures/}}
\DeclareGraphicsExtensions{.pdf,.jpeg,.png,.jpg}

%%   --------------------------------
%%               Theorems
%%   --------------------------------
%\theoremstyle{plain}
%\theoremstyle{definition}
%\floatname{algorithm}{Algorithm}
%\newtheorem{axiom}{Axiom}[section]
%\newtheorem{condition}{Condition}[section]
%\newtheorem{corollary}{Corollary}[section]
%\newtheorem{definition}{Definition}[section]
%\numberwithin{equation}{section}
%\newtheorem{hypothesis}{Hypothesis}
%\newtheorem{lemma}{Lemma}[section]
%\newtheorem{property}{Property}[section]
%\newtheorem{proposition}{Proposition}[section]
%\newtheorem{theorem}{Theorem}[section]
%
%%\renewcommand{\rmdefault}{phv} % Arial
%%\renewcommand{\sfdefault}{phv} % Arial

\definecolor{MyDarkBlue}{rgb}{0 0.08 0.45}
\definecolor{MyDarkOrange}{rgb}{1 0.5 0}
\definecolor{MyDarkGreen}{rgb}{0 0.25 0}
\definecolor{MyDarkTerracota}{rgb}{0.5 0 0}
\definecolor{MyDarkGrey}{rgb}{0.25 0.25 0.25}

\begin{document}
\begin{center}
{\sc EST-25134: Aprendizaje Estad\'istico} -- ITAM\\
\vspace{0.1cm}
Primavera 2017\\
\vspace{0.1cm}
Sesi\'on 01
\end{center}


\def\dsp{\def\baselinestretch{0.7}\large\normalsize}\dsp

%
%	Funciones de distribuci\'on
%
\section*{\normalsize Funciones de distribuci\'on}
\noindent
Traten de identificar el kernel y constante de normalizaci\'on en las siguientes distribuciones:
\begin{itemize}
	\item {\it Distribuci\'on Bernoulli.-} Para $X$ v.a. con soporte en $\mathcal{X}=\{0,1\}$, la funci\'on de masas de probabilidades es,
	\begin{equation}
	f_X(x|\alpha,\beta)=\alpha^{x}(1-\alpha)^{1-x}\boldsymbol{1}_{\{0,1\}}(x),
	\end{equation}
	con $0<\alpha<1$.
	\item {\it Distribuci\'on beta.-} Para $X$ v.a. con soporte en $\mathcal{X}=(0,1)$, la funci\'on de densidad,
		\begin{equation}
		 f_X(x|\alpha,\beta)=\frac{\Gamma(\alpha+\beta)}{\Gamma(\alpha)\Gamma(\beta)}
		 x^{\alpha-1}(1-x)^{\beta-1}\boldsymbol{1}_{(0,1)}(x),
		\end{equation}
	con $\alpha,\beta > 0$.
	\item {\it Distribuci\'on multinomial.-} Para $X$ variable aleatoria con soporte en $\mathcal{X}=\{0,1\}^K$, con $K$ finito y conocido, tenemos que la funci\'on de masas de probabilidades es,
		\begin{equation}
		f_X(x|\alpha_1,\ldots,\alpha_K)=\prod_{k=1}^{K}\alpha_k^{x_k}\boldsymbol{1}_{\{0,1\}^K}(x),
		\end{equation}
		donde $\alpha_k \geq 0$ y $\sum_{k=1}^{K}=1$.
	\item {\it Distribuci\'on Dirichlet.-} Para $X$ v.a. con soporte en $\mathcal{X}=\mathcal{S}_K$ (simplex de dimensi\'on $(K-1)$), de tiene que la funci\'on de densidad es,
		\begin{equation}
		f_X(x|\alpha_1,\ldots,\alpha_K)=\frac{\Gamma(\alpha_0)}{\Gamma(\alpha_1)\cdots \Gamma(\alpha_K)}
		\prod_{k=1}^{K}x_k^{\alpha_k}\boldsymbol{1}_{\mathcal{S}_K}(x),
		\end{equation}
		donde $\alpha_0=\sum_{k=1}^{K}\alpha_k$, con $\alpha_k > 0$.
\end{itemize}
%=======================================================
%		References
%=======================================================
\bibliographystyle{chicago}
\bibliography{References_JCMO}

\end{document}
%
%	-- FIN --